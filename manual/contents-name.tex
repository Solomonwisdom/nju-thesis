\chapter{“目次”和“目录”的区别}

通常我们都将书籍的章节列表称之为“目录”,但根据国家标准《GB/T 7713.1-2006: 学位论文编写规则》
\cite{gbt7713.1-2006}和《GB/T 13417-2009: 期刊目次表》\cite{gbt13417-2009},
以及《南京大学博士(硕士)学位论文编写格式规定(试行)》(参见附录\ref{chap:njureq})中的要求,
论文的章节列表其实应该称为“目次”。

那么“目次”究竟是什么意思?它和“目录”有何区别?

\section{百度百科的解释}

对于目录与目次的区别,百度百科上说明如下\cite{baidu2013muci}:

\begin{quotation}
Contents 目次

Table of Contents 目录

目次是目录的排序,目录是内容章节的具体名称。

目录专著始于刘向父子,到清代 《四库全书总目提要》里按经,史,子,集四类编目,每
一大类又分若干小类,类下分子目,大类前有总序,每一小类前有小序,子目后有案语,序
及案语是用来简述著作源流以及分类理由的。而每小类的后面还附有“存目” ,这存目就是
所谓的目次了 。目录又称为书目,是记录图书的书目名称著者,搜藏与流传情况,内容提
要,评价,真伪辨析等内容的。
\end{quotation}

不过百度百科的解释明显不清楚,且有错误。英语单词``content''在Merriam-Webster
在线版词典中作为名词的解释如下:
\begin{quotation}
a: something contained - usually used in plural;\\
b: the topics or matter treated in a written work;\\
c: the principal substance (as written matter, illustrations, 
or music) offered by a World Wide Web site.
\end{quotation}

显然后两个含义都是第一个含义的引申,即``content''作为名词其本意应该是指“内部包含
之物(something contained)”,后引申为文章的“内容”。而``table of contents''
则有“内容列表”的含义,和“目录”或“目次”相关。

\section{程千帆先生的意见}

程千帆先生反对用“目录”这个词。他说:“我写书时,对于底下的篇目我不用目录两个字的,
因为目是目,录是录,我总是写作目次,写篇目也可以,无论如何不能写目录。”\cite{cheng2008}

程先生的话与传统的目录学知识相关。刘向的《七略》是我国古代最早的全国综合性目录,
虽然今已失传,但由于《汉书·艺文志》是根据《七略》编写的,因此仍然可以知道其大概
体例。

\section{“目录”一词的由来}

《汉书·艺文志》中记载道:

\begin{quotation}
昔仲尼没而微言绝,七十子丧而大义乖。故《春秋》分为五,《诗》分为四,《易》有数家
之传。战国从衡,真伪分争,诸子之言纷然殽乱。至秦患之,乃燔灭文章,以愚黔首。汉兴,
改秦之败,大收篇籍,广开献书之路。迄孝武世,书缺简脱,礼坏乐崩,圣上喟然而称曰:
“朕甚闵焉!”于是建藏书之策,置写书之官,下及诸子传说,皆充秘府。至成帝时,以书颇
散亡,使谒者陈农求遗书于天下。诏光禄大夫刘向校经传诸子诗赋,步兵校尉任宏校兵书,
太史令尹咸校数术,侍医李柱国校方技。每一书已,向辄条其篇目,撮其指意,录而奏之。
会向卒,哀帝复使向子侍中奉车都尉歆卒父业。歆于是总群书而奏其《七略》,故有《辑略》,
有《六艺略》,有《诸子略》,有《诗赋略》,有《兵书略》,有《术数略》,有
《方技略》。今删其要,以备篇辑。
\end{quotation}

其中所说的“每一书已,向辄条其篇目,撮其指意,录而奏之”便是“目录”一词最初的用意了。
即“目”指“条其篇目”,“录”指“条其篇目,撮其指意,录而奏之”。这里采用的是余嘉锡先生在
《目录学发微·何谓目录》中的意见,也就是说“目”单指篇目,而“录”是包含篇目与内容大
意两部分内容的。而后由于袭用的缘故,“录”反而隶属在“目”之下了,于是有篇目而无内容
大意的也可称之为“目录”。再往后,只有书名而无篇目的也可称为“目录”。

《昭明文选·任彦开为范始兴求立太宰碑表(李善注)》引刘欲《七略》称:“《尚书》有青丝
编目录”,可知刘向校书,即已使用“口录”一词。班因《汉书·叙传》中,亦有“爱着目录,略序
洪烈,述艺文志第十”之句,表明早在汉代,“目录”二字已作为一个名词而被加以使用。

\section{“目录”一词的含义}

目录:是指著录一批相关文献,并按照一定次序编排而成的揭示与报道文献的工具。它是联系
文献与用户的桥梁和纽带。是书籍文章的缩影。

目录是目和录的总称。“目”指篇名或书名,“录”是对“目”的说明和编次。前人把“目”与“录”
编在一起,谓之“目录”。

\section{“目录”一词的其他含义}

在现代计算机技术术语中,常常把文件系统中的“directory”称为“目录”,表示
在文件系统树中的非叶子节点。这种用法给“目录”一词带来了新的含义。

\section{结论}

根据“目录”一词的由来,及程千帆先生的意见,我们决定遵循国家标准
《GB/T 7713.1-2006: 学位论文编写规则》\cite{gbt7713.1-2006}和
《GB/T 13417-2009: 期刊目次表》\cite{gbt13417-2009},以及
附录\ref{chap:njureq}中的要求,将``table of contents''称为
“目次”而非“目录”。

